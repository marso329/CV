% -- Encoding UTF-8 without BOM
% -- XeLaTeX => PDF (BIBER)

\documentclass[]{cv-style}          % Add 'print' as an option into the square bracket to remove colours from this template for printing. 
                                    % Add 'espanol' as an option into the square bracket to change the date format of the Last Updated Text

 \usepackage[hidelinks]{hyperref} 
\sethyphenation[variant=british]{english}{} % Add words between the {} to avoid them to be cut 


\begin{document}

\header{Martin}{Söderén}
       {Master of Science in Computer Science and Engineering}        % Your name
%\lastupdated

%----------------------------------------------------------------------------------------
%	SIDEBAR SECTION  -- In the aside, each new line forces a line break
%----------------------------------------------------------------------------------------

\begin{aside}
%
\section{contact}
Rt.~du~Mandement~17
1217 Meyrin
Switzerland
~
+3372 242 73 64
    ~
    \href{mailto:martin.soderen@gamil.com}{martin.soderen@gmail.com}
    \href{https://se.linkedin.com/in/martin-s%C3%B6der%C3%A9n-272002a7}{\mbox{Martin Söderén@Linkedin}}
        \href{https://github.com/marso329}{\mbox{marso329@github}}
\section{languages}
    \mbox{Swedish - mother tongue  }
    English - fluent
    German - basic  
    French - basic
\section{programming}
C 
C++ 
Python   
VHDL          
BASH            
Make               
CMake          
Java      
\LaTeX{} 
% 
\end{aside}

%----------------------------------------------------------------------------------------
%	SKILLS SECTION
%----------------------------------------------------------------------------------------

\section{Skills}
  \vspace{-0.2cm}
\begin{entrylist}
\parbox[t]{3.7cm}{Parallel programming}  \textbf{Designing parallel analysis applications using the most efficient algorithms} \\
 \vspace{\parsep}
 \parbox[t]{3.7cm}{ ~}	\textbf{and synchronization methods for a particular problem} \\
  \parbox[t]{3.7cm}{GPGPU development}  \textbf{Developing real-time digital signal processing analysis applications for} \\
 \vspace{\parsep}
 \parbox[t]{3.7cm}{ ~}	\textbf{GPUs using either CUDA or OpenCL} \\
 \vspace{\parsep}
   \parbox[t]{3.7cm}{Code optimization}  \textbf{Profile and optimize code using all of the available tools} \\
   \vspace{\parsep}
   \parbox[t]{3.7cm}{Control applications}  \textbf{Developing real-time control applications for critical systems} \\
 \vspace{\parsep}
   \parbox[t]{3.7cm}{Linux device drivers}  \textbf{Developing Linux device drivers for PCIe-based high-performance acquisition cards} \\
  \vspace{\parsep}
  \parbox[t]{3.7cm}{Linux administration}  \textbf{Setting up, tuning, and administrating critical Linux systems} \\
  
  \parbox[t]{3.7cm}{UI design}  \textbf{Developing complete user applications using Qt or PyQt} \\
 


\end{entrylist}

%----------------------------------------------------------------------------------------
%	WORK EXPERIENCE SECTION
%----------------------------------------------------------------------------------------

\parskip=1pt 
\section{Experience}

\begin{entrylist}
    \entry
    {2021–now}
    {Staff, CERN, Geneve}
    {Full time}
    {\emph{Continuing my work on acquisition and analysis systems while \\ being responsible for the operation of certain critical systems }}
         \entry
    {2018–2021}
    {Fellow, CERN, Geneve}
    {Full time}
    {\emph{Developing new real-time acquisition and analysis systems for\\ all accelerators. This involved firmware, driver, and analysis application\\ development and the deployment and support of the systems}}
         \entry
    {2016–2017}
    {Technical student, CERN, Geneve}
    {Student program}
    {\emph{Developing applications for analyzing gigabit data-stream in real-time }}
         \entry
    {06–08 2016}
    {Linköping University, Linköping}
    {Summer job}
    {\emph{Software developer for AIICS}}
         \entry
    {08–12 2015\\08–12 2014}
    {Linköping University, Linköping}
    {Part time}
    {\emph{Course assistant in Perspectives in Computer\\ Science and Computer Engineering}}
  
  \entry
    {06–08 2015}
    {SAAB, Linköping}
    {Summar job}
    {\emph{Software developer}}
    
   \entry
    {06–08 2015\\06–08 2014 }
    {Linköping University, Linköping}
    {Summer job}
    {\emph{Software developer}}
   
       \entry
    {11–12 2014}
    {Linköping University, Linköping}
    {Part time}
    {\emph{Course assistant in Introduction to design and product development}}

    
  \entry
    {06–08 2014\\06–08 2013\\06–08 2011 }
    {Silex Microsystems, Stockholm}
    {Summer job}
    {\emph{Process operator working with photolithography}}
    

\end{entrylist}

%\parskip=0pt 
\section{Education}

\begin{entrylist}
  \entry
    {2021}
    {Expert VHDL}
    {DOULOS, CERN}
    { }
  \entry
    {2018}
    {Comprehensive VHDL for FPGA Design}
    {DOULOS, CERN}
    { }
  \entry
    {2018}
    {CAS - Introduction to Accelerator Physics}
    {CERN}
    { }
        \end{entrylist}
    \begin{entrylist}
    
  \entry
    {2017}
    {Thematic CERN school of Computing}
    {CERN}
    { }
  \entry 
    {2011–2017}
    {M.Sc in Computer Science and Engineering}
    {Linköping University} 
    { }
\end{entrylist}

\begin{entrylist}


  \entry
    {2009–2010}
    {Engineering preparatory year}
    {KTH Royal Institute of Technology}
    {}
  \entry
    {2006–2009}
    {Social studies with major in economics}
    {Blackebergs gymnasium}
    {}
\end{entrylist}




%----------------------------------------------------------------------------------------
%	AWARDS SECTION
%----------------------------------------------------------------------------------------

\section{Publications}
\begin{entrylist}  
           \entry
    {06-2022}
    {Reconstruction of transverse phase space from transverse feedback data for real time extraction of vital LHC machine parameters}
    {CERN, Geneva}
    {\emph{}{IPAC paper}}
    \entry 
    {06-2022}
    {Digital low-level RF system for the CERN Linac3 accelerator}
    {CERN, Geneva}
    {\emph{}{IPAC paper}}   
    \entry
    {08-2020}
    {ADTObsBox to catch instabilities}
    {CERN, Geneva}
    {\emph{}{MCBI proceeding}}   
   
      \entry
    {11-2019}
    {Low Latency, Online Processing of the High-Bandwidth Bunch-by-Bunch \\ Observation Data from the Transverse Damper Systems of the LHC}
    {CERN, Geneva}
    {\emph{}{CHEP paper}}  
    
      \entry
    {10-2017}
    {Online coupling measurement and correction throughout the LHC cycle}
    {CERN, Geneva}
    {\emph{}{ICALEPCS paper}}   

  
      \entry
    {09-2017}
    {ADT And OBSBOX In LHC Run 2, Plans For LS2}
    {CERN, Geneva}
    {\emph{}{LHC Operations procedding}}
       
      \entry
    {09-2017}
    {Online Transverse Beam Instability Detection in the LHC\\High-Throughput Real-Time Parallel Data Analysis}
    {CERN, Geneva}
    {\emph{}{Master thesis}}

      \entry
    {05-2017}
    {\href{http://accelconf.web.cern.ch/AccelConf/ipac2017/papers/mopab117.pdf}{Usage of the Transverse Damper Observation Box for \\High Sampling Rate Transverse Position Data in the LHC}}
    {CERN, Geneva}
    {\emph{}{IPAC paper}}

  \entry
    {05-2017}
    {\href{http://accelconf.web.cern.ch/AccelConf/ipac2017/papers/mopab117.pdf}{Online Bunch by Bunch Transverse Instability Detection in LHC}}
    {CERN, Geneva}
    {\emph{}{IPAC paper}}
  

  \entry
    {05-2015}
    {\href{http://liu.diva-portal.org/smash/get/diva2:844414/FULLTEXT01.pdf}{Development of a quadratic programming \\ solver for model predictive control}}
    {SAAB, Linköping}
    {\emph{}{Bachelor thesis}}
   
    
\end{entrylist}

%----------------------------------------------------------------------------------------
%	INTERESTS SECTION
%----------------------------------------------------------------------------------------
\section{Private Projects}

\begin{entrylist}
    \entry
    {2022}
    {AVX512VectorLib}
    {}
    {\emph{\href{https://github.com/marso329/AVX512VectorLib}{A vector operations library that utilizes AVX512 instructions }}}
    \entry
    {2020}
    {HDLCompiler}
    {}
    {A VHDL/Verilog compiler for real-time semantic analysis }
    
    \entry
    {09-2019}
    {Super Plastic Synchrotron}
    {}
    {\emph{\href{https://github.com/marso329/SuperPlasticSynchrotron}{A working model of a particle accelerator for CERN open days}}}
    
    
                \entry
    {09-2017}
    {ConcurrentDataSharer}
    {} 
    {\emph{\href{https://github.com/marso329/ConcurrentDataSharer}{Publish-subscribe pattern middleware for easy data sharing}}}
    
    
  \entry
    {06-2016}
    {FPGAComputer}
    {}
    {\emph{\href{https://github.com/marso329/FPGAComputer}{Complete computer implemented in VHDL}}}

\end{entrylist}

\begin{entrylist}


    
      \entry
    {06-2016}
    {3DFileManager}
    {}
    {\emph{\href{https://github.com/marso329/3DFileManager}{A 3D file manager implemented in OpenGL}}}
   
    
    
                   \entry
    {12-2015}
    {matLib}
    {}
    {\emph{\href{https://github.com/marso329/matLib}{A barebone matrix/vector library }}}
\end{entrylist}
\section{Interest}
Mechanical Engineering, Robotics, Snowboarding, Swimming, Traveling, Hiking, and Motorcycles.
\end{document}