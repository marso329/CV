% -- Encoding UTF-8 without BOM
% -- XeLaTeX => PDF (BIBER)

\documentclass[left=2.5cm,top=2cm,right=2.5cm,bottom=2.5cm]{cv-style}   
       % Add 'print' as an option into the square bracket to remove colours from this template for printing. 
                                    % Add 'espanol' as an option into the square bracket to change the date format of the Last Updated Text
 \usepackage[hidelinks]{hyperref} 
\sethyphenation[variant=british]{english}{} % Add words between the {} to avoid them to be cut 
%\RequirePackage[left=2.5cm,top=2cm,right=2.5cm,bottom=2.5cm,nohead,nofoot]{geometry}
\geometry{left=2.5cm,right=2.5cm}
\begin{document}

\header{Martin}{Söderén}
       {Master of Science in Computer Science and Engineering}        % Your name
\vspace{10mm}
\\
Dear Sir or Madam
\\
\\
I am a computer scientist that takes pride in what I do and always wants to deliver professional engineering solutions. After working at CERN for four years and spending a lot of time with the talented people who are pushing technology to its limits to further expand our knowledge of the universe I would like to continue with this great task.

One of my greatest strengths is my knowledge of many different fields and my capacity to apply this to the different parts of engineering projects. This allows me to be a part of many different computer science-related projects and to have an active role from start to finish. This could either mean that I am performing all of the tasks by myself or that I am working as a part of a group. I also enjoy taking an organizing role where I coordinate the different people involved. I also make sure to be informed about the latest innovations in the different fields such that I have a good knowledge base to solve problems to which no predefined solutions might exist.

During my time at CERN, I was responsible for many different projects. The main achievement was the development of the next generation observation system of the transverse beam position in the LHC. This was a complex project which consisted of a multitude of different tasks such as:
\begin{itemize}
 \setlength{\itemsep}{3pt}
    \setlength{\parskip}{0pt}
    \setlength{\parsep}{0pt} 
\item Selecting and purchasing new PCIe based acquisition cards
\item Developing new firmware for receiving high-bandwidth data streams
\item Developing high-performance Linux device drivers for transferring the data to the host
\item Developing high-level libraries to communicate with the driver
\item Developing analysis applications to analyze the transverse oscillation which utilized parallel computing
\item Selecting and purchasing high-performance servers
\item Overseeing the installation of racks, network, power, timing distribution and, fiber-optics
\item Setting up and administrating the servers
\item Testing and verifying the reliability and performance of the system 
\end{itemize}
During my time at CERN, I have been involved in a multitude of different projects which required a knowledge of digital signal processing on either CPUs or FPGAs together with a knowledge of transverse particle beam dynamics. 
A none-comprehensive list of projects that I have taken part in:
\begin{itemize}
 \setlength{\itemsep}{3pt}
    \setlength{\parskip}{0pt}
    \setlength{\parsep}{0pt} 
\item Maintaining and upgrading the older generation of the ADTObsBox system
\item Developing control applications for the transverse feedback system in the LHC
\item Upgrade of PS transverse feedback 
\item Upgrade of Booster transverse feedback
\item Developing a unified control application for all transverse feedback systems in the injector chain
\item Developing the AD Schottsky observation system
\item Developing the online coupling correction system in the LHC
\item Developing the ADT instability detection system
\end{itemize}

My role in the different projects ranges from being the sole contributor to working in a larger group with colleagues from different sections. I have also assisted in the commissioning and operations of the various systems.

I have always enjoyed teaching, at university, I was working for the computer science department to create a project for students as a part of a course in introductory programming\footnote{https://www.ida.liu.se/~TDDE25/}. This goal of the project was to create a robotic dog and while creating the project I had to select the hardware, create the software libraries, write the documentation, and prepare the instructions for the students. I was also the supervisor for up to 36 students over the course of three years. During this course I introduced students to GUI programming, debugging, object-oriented programming, and code versioning tools to name a few. This passion has continued at CERN and I am currently working on setting up a collaboration agreement with LiU University in Sweden and CERN. The purpose of this agreement is to advertise to Swedish students that it is possible to come here for a period of time and write a thesis. I would enjoy working as a technical supervisor for these students.

During my time at university, I was also a part of a mechanical engineering club\footnote{http://www.m-verkstan.se/} where I learned a lot about mechanical engineering. This allowed me to be a TA in a few design courses\footnote{http://www.mechanics.iei.liu.se/edu\_ug/tmpm01/} where I supervised groups of students in their process of designing different mechanical appliances. I was also vice-president and head of the electronics group in the university formula student team \footnote{http://www.liuformulastudent.se/} where we designed, manufactured, and competed with a full-size race car. This passion for mechanical engineering has continued at CERN where I from time to time have helped out colleagues with CAD and manufacturing of components required for the installation of electronics in the accelerators. 

My field of expertise is parallel and distributed computing using either CPUs, GPUs, or FPGAs. However, I want to understand the complete chain of components that make up a system. This means understanding how the electronics are implemented, how the data is processed in the integrated circuits, how the data is transferred to a host computer, how the driver communicates with the hardware, and how the applications read the data from the driver. By understanding this chain, I can improve the whole chain to operate as close to optimal as possible. This is of course done in collaboration with the people who will take advantage of the system. Discussing the requirements and features with them is an interesting part of the process and seeing how the different systems help people with their work at CERN is enriching. 

I am applying for a position at CERN because I deeply enjoy working for CERN and would like to continue delivering excellent engineering solutions that will continue our quest for knowledge while continuing to both learn and teach. I am applying for this position because it combines many of the different subjects that I find interesting such as firmware development, driver development, digital signal processing, and high performance parallel analysis application development. I hope to hear from you soon.

Sincerely,\\
\\
\\
Martin Söderén
\end{document}
