% -- Encoding UTF-8 without BOM
% -- XeLaTeX => PDF (BIBER)

\documentclass[left=2.5cm,top=2cm,right=2.5cm,bottom=2.5cm]{cv-style}   
       % Add 'print' as an option into the square bracket to remove colours from this template for printing. 
                                    % Add 'espanol' as an option into the square bracket to change the date format of the Last Updated Text
 \usepackage[hidelinks]{hyperref} 
\sethyphenation[variant=british]{english}{} % Add words between the {} to avoid them to be cut 
%\RequirePackage[left=2.5cm,top=2cm,right=2.5cm,bottom=2.5cm,nohead,nofoot]{geometry}
\geometry{left=2.5cm,right=2.5cm}
\begin{document}

\header{Martin}{Söderén}
       {Master of Science in Computer Science and Engineering}        % Your name
\vspace{10mm}
\\
Dear Sir or Madam
\\
\\
I am a computer scientist that takes pride in what I do and always wants to deliver professional engineering solutions. After working at CERN for six years and spending a lot of time with the talented people who are pushing technology to its limits to further expand our knowledge of the universe I would like to continue with this great task.

One of my greatest strengths is my knowledge of many different fields and my capacity to apply this to the different parts of engineering projects. This allows me to be a part of many different computer science-related projects and to have an active role from start to finish. This could either mean that I am performing all of the tasks by myself or that I am working as a part of a group. I also enjoy taking an organizing role where I coordinate the different people involved. I also make sure to be informed about the latest innovations in the different fields such that I have a good knowledge base to solve problems to which no predefined solutions might exist.

During my time at CERN I have worked with a multitude of different science fields ranging from digital signal processing on FPGAs to UI design for applications used in the operation of the machines. However, my field of expertise is in parallel and distributed computing using either CPUs or GPUs.
I'm also an avid CAD user and I have been fascinated in mechanical engineering for a long time. During my time at university, I was a part of a mechanical engineering club\footnote{http://www.m-verkstan.se/} where I learned a lot about mechanical engineering. This allowed me to be a TA in a few design courses\footnote{https://studieinfo.liu.se/kurs/tmpm05/ht-2021} where I supervised groups of students in their process of designing different mechanical appliances.

It is rewarding to develop optimized code, it is even more rewarding when you can visualize the result in a beautiful way. This is why I find parallel computing combined with GUI development fascinating, especially when combined with OpenGL or Vulkan to provide a detailed and smooth user experience. 

I'm currently looking for an environment where I can combine my fascination for efficient C++ programming, mathematics, CAD and mechanical engineering. When reading about COMSOL it feels like just such a place and where I can be a valuable member of your team.

I'm currently living in Geneva, Switzerland but I'm a Swedish citizen and I grew up in Bromma, Stockholm. If you were to hire me, I would move back to Stockholm as soon as possible.

I hope to hear from you soon.


Sincerely,\\
\\
\\
Martin Söderén
\end{document}
